\documentclass[slidestop,compress,mathserif]{beamer}

\usepackage[bars]{beamerthemetree} % Beamer theme v 2.2
\setbeamertemplate{navigation symbols}{}
\setbeamertemplate{theorems}[numbered] 

\setbeamercolor{frametitle}{fg=white,bg=white}
\setbeamercolor{title}{fg=blue!60!black,bg=blue!6!white}
\usetheme{PaloAlto}


\usecolortheme[named=blue]{structure}
\usepackage{fontawesome}
% \definecolor{mypurple}{rgb}{.49,0,98}
% \setbeamercolor{sidebar}{bg=red!85!black}
% \setbeamercolor{frametitle}{bg=green!70!black}
% \setbeamercolor{logo}{bg=mypurple}
% \setbeamercolor{sidebar}{bg=red!85!black}
% \setbeamercolor{frametitle}{bg=green!70!black}
% \setbeamercolor{logo}{bg=mypurple}


\usepackage[utf8]{inputenc}


\usepackage{amssymb}
\usepackage{enumerate}%\usepackage{paralist}
\usepackage{bm}%bold math
\usepackage{amsmath}%math formulas
\usepackage{amsthm}%ems,Proofs
\usepackage{yhmath}%dots
\usepackage{fontenc}
\usepackage{wasysym}%Various math symbols
\usepackage{MnSymbol}
\usepackage{color}
\usepackage{graphicx}
\usepackage{tabularx}
\usepackage{epstopdf}
\usepackage{url}
\usepackage{hyperref}
\usepackage{booktabs}
\usepackage[font=small,labelfont={tiny},{bf}]{caption}
\usepackage{listings}

\usepackage{xcolor}
\definecolor{myred}{HTML}{CA3C32}
\definecolor{mygreen}{HTML}{3A9746}
\definecolor{mypurple}{HTML}{9159A2}


\usepackage{subcaption}
\usepackage{amsmath}
\usepackage{graphicx} % Allows including images
\usepackage{booktabs} % Allows the use of \toprule, \midrule and \bottomrule in tables

%%%new commands
\newcommand\gre\relax
\newcommand\eng\relax
\newcommand{\norm}[1]{\lVert #1 \rVert }
\newcommand{\bb}[1]{\mathbb{#1}}
\newcommand{\la}{\lambda}
\newcommand{\isod}{\Leftrightarrow}
\newcommand{\ceil}[1]{\left\lceil #1 \right\rceil}

%%%
\usepackage{hyperref}
\usepackage{ragged2e} % provides the \justifying command
\usepackage{nameref}


%%
\usepackage{tikz}
\usetikzlibrary{calc,matrix,arrows.meta}
\usepackage{ragged2e}
\renewcommand{\raggedright}{\leftskip=0pt \rightskip=0pt plus 0cm}

\beamerdefaultoverlayspecification{<+->} %stepwise uncovering
\setbeamercovered{transparent}
\usepackage{hyperref}


%---------------------------------------------------------------------------------------------------
%   TITLE PAGE
%---------------------------------------------------------------------------------------------------

\title[]{Julia Programming Language: \\ Guide \&
Reactive Applications}

\author[]{ 
} 
\date{}

\begin{document}
\section{Poster Introduction}

\begin{frame}
	\vskip -0.5cm
	\titlepage % Print the title page as the first slide
	\vspace{-2.5cm}
	\hspace{60pt}	\href{https://www.math.uoa.gr}{\includegraphics[scale=0.8]{ekpa_logo.pdf}}
	\begin{center}
	   \textcolor{blue}{\scriptsize \textbf{Julia Team of Mathematics NKUA \\
                     National and Kapodistrian University of Athens } }
                      \hyperlink{target2}{\beamergotobutton{\href{https://github.com/nkuamath-julia/NASCA23}{\faicon{github} : github.com/nkuamath-julia/NASCA23}}}
	\end{center}
    
\end{frame}


%---------------------------------------------------------------------------------------------------
%   PRESENTATION SLIDES
%---------------------------------------------------------------------------------------------------

%---------------------------------------------------------------------------------------------------
%   SLIDE 1
%---------------------------------------------------------------------------------------------------
\section{Julia in a Nutshell}
\begin{frame}
\frametitle{Julia in a Nutshell}


\begin{center}
    \fontsize{10pt}{7.2}\selectfont \textbf{Release of Julia 1.0, 8 August 2018} \\ 
    \fontsize{9pt}{7.2}\selectfont Created by: Jeff Bezanson, Stefan Karpinski, Viral B. Shah & Alan Edelman\\
\end{center}
\raggedright \fontsize{8pt}{7.2}\selectfont \vfill Quote from the creators:\\
\begin{center}
\textcolor{blue}{\fontsize{10.3pt}{3}\selectfont
“We want a language that’s \textbf{open source}, with a liberal
license.\\
We want the \textbf{speed} of C\\
 with the \textbf{dynamism} of Ruby.\\
We want a language that’s \textbf{homoiconic},\\
 with true \textbf{macros} like Lisp,\\
 but with obvious, familiar \textbf{mathematical notation} like Matlab.\\
We want something as usable for \textbf{general programming} as Python,\\
 as easy for \textbf{statistics} as R,\\
 as natural for \textbf{string processing} as Perl,\\
 as powerful for \textbf{linear algebra} as Matlab,\\
 as good at \textbf{gluing programs} together as the shell.\\
 Something that is \textbf{dirt simple} to learn,\\
 yet keeps the most \textbf{serious hackers} happy.\\
 We want it \textbf{interactive} and we want it \textbf{compiled}.”
 }
 \end{center}
\end{frame}


%---------------------------------------------------------------------------------------------------
%   SLIDE 2
%---------------------------------------------------------------------------------------------------
\section{Content} 

%---------------------------------------------------------------------------------------------------

\begin{frame}{Fully justified}
\frametitle{Poster's Content}

\raggedright
%\vfill
\vfill
	\begin{itemize}
		{\footnotesize
		\setlength{\itemsep}{5mm}
        \item Introduction \& Guide of Julia Language
		\item The accuracy of different expressions representing equal quantities
		\item Interactive program that converts numbers from decimal to $b$-ary
		\item Interactive program that prints binary rational numbers within a given range
		\item Benchmarks of algorithms which find wheter a number is prime or not 
%		\item Calculation of pivot patterns in Hadamard matrices using Gaussian Elimination with complete pivoting,
		\item  Julia's efficiency in the Newton-Raphson Method \& matrix computations, benchmarking it against Matlab and Python
        \item Lagrange \& Newton Interpolation
	 	 }
	\end{itemize}
	\vfill
\end{frame}

%---------------------------------------------------------------------------------------------------
%   ENDING 
%---------------------------------------------------------------------------------------------------

\section{Ending}
\begin{frame}[fragile]{}
%\frametitle{General Information}
%
    \begin{center}
        \fontsize{20pt}{7.2}\selectfont \textbf{Thank you for your time!}\\\vfill
        \fontsize{15pt}{7.2}\selectfont \textcolor{blue}{We would love to see you on the coffee breaks!}
    \end{center}

\vfill
%\vspace{2cm}
\begin{center}
The future seems \textcolor{myred}{B} \textcolor{mygreen}{r} \textcolor{mypurple}{i} \textcolor{myred}{g} \textcolor{mygreen}{h} \textcolor{mypurple}{t}
\begin{figure}
\includegraphics[scale=0.07]{julia.png}
\end{figure}
\end{center}\vfill
\begin{center}
	\textcolor{blue}{\scriptsize \textbf{Julia Team of Mathematics NKUA \\
                 National and Kapodistrian University of Athens\\ } }

 \hyperlink{target2}{\beamergotobutton{\href{https://github.com/nkuamath-julia/NASCA23}{\faicon{github} : github.com/nkuamath-julia/NASCA23}}}
\end{center}
\vfill

\end{frame}
\end{document}
